%	\subsection{Decade Counter through Flip Flops}
%
Open the blink program.  You will see the following
%\lstinputlisting[language=C]{./codes/blink}
\lstinputlisting[language=C]{./codes/blink/src/main.cpp}
\begin{problem}
	Connect the digital pin 13 of the arduino to the {\em dot} pin of the display. Execute the Blink program.
\end{problem}
\begin{problem}
Change the delay to 500 ms in the program and execute.  What do you observe?
\end{problem}
%	
The 7474 IC in Fig. \ref{fig:7474} has two D flip flops.  The D pins denote the input and the Q pins denote the output. CLK denotes the clock input.
%
\begin{figure}[!h]
\begin{center}
\resizebox {\columnwidth} {!} {
\input{./figs/7474.tex}
}
\end{center}
\caption{}
\label{fig:7474}
\end{figure}
%

%%
%\begin{figure}[!h]
%\begin{center}
%\includegraphics[width=\columnwidth]{./figs/7474IC}
%\end{center}
%\caption{}
%\label{fig:7474IC}
%\end{figure}

%\begin{center}
	%\includegraphics[scale=1]{7474IC}
%\end{center}
\begin{problem}
Connect the Arduino, 7447 and the two 7474 ICs according to Table \ref{fig:ff_ard_pin}.
\end{problem}
%
\begin{problem}
Connect the 7447 IC to the seven segment display.
\end{problem}
%\begin{problem}
%Connect the D2-D5 pins of the arduino to the Q pins of the two 7474 ICs. Use the D2-D5 pins as Arduino input.
%\end{problem}
%\begin{problem}
%Connect the Q pins to IC 7447 Decoder as input pins.  Connect the 7447 IC to the seven segment display.
%\end{problem}
%\begin{problem}
%Connect the D6-D9 pins of the arduino to the D input pins of two 7474 ICs. Use the D6-D9 pins as Arduino output.
%\end{problem}
%\begin{problem}
%Connect pin 13 of the Arduino to the CLK inputs of both the 7474 ICs.
%\end{problem}
%\begin{problem}
%Connect pins 1, 4, 10 and 13 of both 7474 ICs to 5V.
%\end{problem}
\begin{problem}
Use the logic for the counting decoder in Problem \ref{prob:7447_decade} to implement the decade counter. You may refer to Fig. \ref{fig:decade_counter} to understand the
functioning of a decade counter.

\end{problem}
%
%
\begin{figure}[!h]
\begin{center}
\resizebox {\columnwidth} {!} {
\input{./figs/decade_counter.tex}
}
\end{center}
\caption{}
\label{fig:decade_counter}
\end{figure}
%

%\subsection{Ripple Counter}
%%
%\begin{problem}
%Using the Arduino for combinational logic and flip-flops, implement the ripple decade counter.
%\end{problem}
%\begin{problem}
%Using the D2-D5 pins as input and D6-D9 pins as output, write a program to implement the logic functions in Problem \ref{seq_decoder} and execute the program.  Comment.
%\end{problem}


%\begin{problem}
%Draw the state transition diagram for the decade counter.  Number the states from 0-9
%\end{problem}
%\begin{problem}
%Draw the state transition table that has present and next state values in binary.
%\end{problem}

