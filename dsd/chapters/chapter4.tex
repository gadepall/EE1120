%\subsection{Decade Counter through Flip Flops}
%
%Consider the truth table for the segment $a$
%\input{a_out}
%%
%The boolean logic function for segment $a$ can be simplified using a Karnaugh map (K-map) 
%\begin{center}
%\includegraphics[scale=1]{kmap}
%\end{center}
%and results in the function
%%
%\begin{equation}
%\label{kmap_fn}
%a = A + C + \bar{B}\bar{D}+ BD
%\end{equation}
\begin{problem}
Use  K-map generated expressions for the counting decoder in Table \ref{table:counter_decoder}.
\end{problem}
%
%
\begin{problem}
Use  K-map generated expressions for segments $a-f$  to drive the display.
\end{problem}
%
%\begin{problem}
%Implement the 3-8 decoder using Arduino.  Verify your results.
%\end{problem}
%%
%\begin{problem}
%Implement the 4-1 MUX using Arduino.  Verify your results.
%\end{problem}
%%
%\begin{problem}
%Implement the 8-3 encoder.  Verify your results.
%\end{problem}
%%
%\begin{problem}
%Implement the 1-4 DEMUX using Arduino.  Verify your results.
%\end{problem}
%%
%\begin{problem}
%Implement 4 bit addition, subtration and multiplication.
%\end{problem}
%%
%\begin{problem}
%Implement 4 bit division.
%\end{problem}


%\begin{problem}
	%\label{seq_decoder}
%Using a K-map, obtain the logic functions for the above transition table. Refer to Problem \ref{counter_dec}
	%\end{problem}	

