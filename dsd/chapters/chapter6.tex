%\subsection{Decade Counter through Flip Flops}
%
%Consider the truth table for the segment $a$
%\input{a_out}
%%
%The boolean logic function for segment $a$ can be simplified using a Karnaugh map (K-map) 
%\begin{center}
%\includegraphics[scale=1]{kmap}
%\end{center}
%and results in the function
%%
%\begin{equation}
%\label{kmap_fn}
%a = A + C + \bar{B}\bar{D}+ BD
%\end{equation}
\begin{problem}
Implement the decade counter using a for loop.
\end{problem}
%
%
%
\begin{problem}
Implement decimal to binary conversion using a loop in Problem \ref{prob:first_code}.
\end{problem}
%
\begin{problem}
Implement decimal to binary conversion using a function.
\end{problem}
%
\begin{problem}
Implement the delay() using millis().
\end{problem}
%
\begin{problem}
Drive 6 displays using the Arduino to build a clock.
\end{problem}
%



%\begin{problem}
	%\label{seq_decoder}
%Using a K-map, obtain the logic functions for the above transition table. Refer to Problem \ref{counter_dec}
	%\end{problem}	

