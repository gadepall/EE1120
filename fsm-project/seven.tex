\documentclass[12pt,onecolumn]{IEEEtran}
\usepackage[latin1]{inputenc}
%\usepackage{graphicx}
%\usepackage{iithhw}
\usepackage{iithquiz}
\usepackage{amsmath}
\usepackage{amscd}
\usepackage{amsfonts}
\usepackage{amssymb}
\usepackage{amsthm}
\usepackage{mathrsfs}
\usepackage{stfloats}
\usepackage{listings}



    \usepackage{longtable}                                        %%
    \usepackage{calc}                                             %%
    \usepackage{multirow}                                         %%
    \usepackage{hhline}                                           %%
    \usepackage{ifthen}                                           %%

%\usepackage{tfrupee}
%\usepackage{multicol}
\usepackage{array}
\DeclareMathOperator*{\Res}{Res}
%\DeclareMathOperator*{\pe}{\Pr}
%\usepackage{amsmath, verbatim,url,graphicx,pxfonts,setspace,fancyhdr}

\newtheorem{theorem}{Theorem}[section]
\newtheorem{proposition}{Proposition}[section]
\newtheorem{lemma}{Lemma}[section]
%\newtheorem{example}{Example}[section]
\newtheorem{example}{Example}
\theoremstyle{definition}
\newtheorem{problem}{Problem}
\newtheorem{cprob}{Challenging Problem}
\newtheorem{computer}{Computer Exercise}
\newtheorem{definition}{Definition}[section]
\newtheorem{algorithm}{Algorithm}[section]
\theoremstyle{remark}
\newtheorem{rem}{Remark}
\def\inputGnumericTable{}                                 %%

\providecommand{\abs}[1]{\left\lvert#1\right\rvert}
\providecommand{\res}[1]{\Res\displaylimits_{#1}} 
\providecommand{\norm}[1]{\lVert#1\rVert}
\providecommand{\mtx}[1]{\mathbf{#1}}
\providecommand{\mean}[1]{E\left[ #1 \right]}
\providecommand{\fourier}{\overset{\mathcal{F}}{ \rightleftharpoons}}
%\providecommand{\hilbert}{\overset{\mathcal{H}}{ \rightleftharpoons}}
\providecommand{\system}{\overset{\mathcal{H}}{ \longleftrightarrow}}
%\newcommand{\solution}[2]{\textbf{Solution:}{#1}}
\newcommand{\solution}{\noindent \textbf{Solution: }}
\providecommand{\pr}[1]{\ensuremath{\Pr\left(#1\right)}}
\providecommand{\qfunc}[1]{\ensuremath{Q\left(#1\right)}}
\providecommand{\sbrak}[1]{\ensuremath{{}\left[#1\right]}}
\providecommand{\lsbrak}[1]{\ensuremath{{}\left[#1\right.}}
\providecommand{\rsbrak}[1]{\ensuremath{{}\left.#1\right]}}
\providecommand{\brak}[1]{\ensuremath{\left(#1\right)}}
\providecommand{\lbrak}[1]{\ensuremath{\left(#1\right.}}
\providecommand{\rbrak}[1]{\ensuremath{\left.#1\right)}}
\providecommand{\cbrak}[1]{\ensuremath{\left\{#1\right\}}}
\providecommand{\lcbrak}[1]{\ensuremath{\left\{#1\right.}}
\providecommand{\rcbrak}[1]{\ensuremath{\left.#1\right\}}}
%\providecommand{\rbrak}[1]{\ensuremath{\left. #1\right \}}}
%\providecommand{\curly}[1]{\ensuremath{\left\{#1\right\}}}

\bibliographystyle{IEEEtran}

\begin{document}

\title{
\logo{Finite State Machine}{}{EE 1120}{Digital System Design}
}
\maketitle
\vspace{-1cm}

%The following cirucit has two outputs $Q$ and $QN$.  The inputs are $R$ and $S$.  
%The following circuit comprises of two D-latches in cascade. C is the enable for the latch. The two waveforms shown below are the D input to the first latch and a periodic waveform CLK. (I know that it doesn't look like one \includegraphics[scale=0.1]{smiley}).  
The following car was manufactured by the Ford corporation in 1965. Note that the first commercial transistor was produced by Texas Instruments in 1954.  
\begin{enumerate}
\item If you wanted to take a LEFT turn, first LA would glow, then LB and then LC and then IDLE (all lights off) according to a clock. This process would continue as long as the LEFT signal was active. 
\item  The process is similar for RIGHT, where the lights on the right side RA, RB and RC, would blink accordingly. 
\item  If the driver is confused, there can also be a HAZ (hazard) signal, where all lights blink at the same time.  This state is similar to the parking lights that we have in contemporary cars.
\end{enumerate}
The objective is to design a circuit for controlling the car lights according to the given information.
\begin{center}
\includegraphics[scale=0.1]{car}
\end{center}

\begin{enumerate}
\item List the input signals for the circuit that you wish to design.
\item List all possible states that you are likely to encounter and assign binary numbers to these states
\item Draw the state transition diagram for the circuit.
\item How many flip-flops would you require?
\item Prepare the transition table for the circuit
\item Write the transition equations for the circuit
\item Implement the final circuit using D flip-flops
\end{enumerate}
Such a circuit is also known as a state machine. 

%


\end{document}
